\chapter{Execution}
\section{Boost converter}
To examine the characteristics of the puls generator, the potentiometer on the circuit board is first set completly
counterclockwise. The power supply is turned on, so that the duty cycle (in \%) and the output voltage can be ritten from
the LCD. The two values are noted and the potentiometer is turned up until the value for duty cycle has increased by 10 \%.
Again the values are noted. This process is repeated until the potentiometer is turned completely clockwise.
%
\section{Avalanche pulse generator}
The potentiometer is set back to the fully counterclockwise position. Now the oscilloscope is needed and therefore
switched on. The output of the puls generator gets connected with the Oscilloscope via a short coaxial cable. The
potentiometer is slowly turned up until a pulse appears on the oscilloscope. The display is adjusted so that the signal
can be read easily. A photograph is taken for documentation purposes and the voltage shown on the LCD is noted. After
that, the voltage is set to \(U = \SI[]{75}[]{V}\) and a second photograph is taken from the screen.
%
\section{Signal propagation}
\subsection{Propagation time}
For this experiment a T-piece is inserted between the oscilloscope and the pulse generator. Therefore the T-piece is
connected to channel 1 of the oscilloscope and the short coaxial cable from the pulse generator is plugged into the
T-piece. A second T-piece is connencted to channel 2 of the oscilloscope. One end of the T-piece in channel 2 is terminated
with the $50\ \Omega$ terminal resistor, meentioned in the set-up chapter. There is one open end left on each T-piece.
The three different coaxial cables get connected one after another to these ends. For each cable, there are two pulses
shown on the oscilloscope. With the curser function of the oscilloscope, the time delay between the pulses are mesaured
and noted.
%
\subsection{Cable characteristics}
To investigate the cable characteristics, the length of the three cables given is measured with the tape measure.
Afterwards they are connected one after another with the T-piece at channel 1. Channel 2 ist not needed during this
measurement. For each cable two photographs are taken from the oscilloscopes screen. The first photograph with an open
end, the secend with an short-circuited end. To short-circuit the end of the coaxial cable, a screwdriver is used. With
the cursor function, the propagation time $\tau_0$ (for open end) and $\tau_s$ (for short-circuited end) are read from
the screen and noted.\\
Now, the termination box is connencted to the open end of the cable. The potentiometer on the termination box is rotated
while looking at the oscilloscopes screen. Once the oscilloscope shows a minimum amplitude of the reflected pulse, the
seting of the potentiometer is not changed anymore. The termination box is removed and then connencted to the multimeter.
The multimeter is set so that the resistance of the termination box can be read from it. This procedure is repeated for
the other remaining cables.
%
\section{Time Domain Reflectometry}
The cable with unkown length and an internal fault is now connected to the T-piece at channel 1. The first cursor of the
oscilloscop is set to the origin pulse. The second cursor is first set to the pulse of the reflection of the internal
fault. The time delay is noted. Then the second cursor is set to the pulse of the reflection of the cables end. The time
delay is noted again.