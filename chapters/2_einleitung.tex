\chapter{Introduction}
%
\section{Terms and Definitions}
    \subsection*{Transmission Line}\label{sec:transmissionLine}
    %
    In there simplest form cables are made out of a conducting material to transport electrical energy or signals from point
    A to point B. The higher the frequency of the signal to transmit, the less the wave nature of can be neglected.
    %
    \subsection*{Characteristic Impedance, Velocity Factor and Propagation Speed}
    %
    Characteristic impedance: The impedance, when connected to a transmission line, suppresses any reflections and standing
    waves \cite{ATIS.AmericanTelecomStandards2001}.
    Velocity factor: Relative signal propagation speed inside a transmission line expressed as percent of speed of light.
    Propagation speed: The absolute speed at which a signal propagates through a medium.
    %
    \subsection*{Time Domain Reflectometry}
    %
    A method to inspect properties of a transmission line i.e. length, characteristic impedance and velocity factor as
    well as the presence, nature and location of defects.
    %
    \subsection*{Avalanche Pulse Generator}
    %
    A circuit to generate ultra short pulses on a scale of picoseconds. Its main working principle abuses the avalanche
    breakdown of a transistor across the collector-emitter line. The breakdown voltage is usually much higher than the
    voltages during normal operation.
    %
    \subsection*{Boost Converter}
    %
    A circuit capable of \textit{boosting} a constant current input voltage to a much higher output voltage by repeatedly
    switching an inductor on and off. The fly back voltage induced by the break down of the magnetic field gets stored in
    a capacitance and forms the voltage at the output terminals.
    %
    \subsection*{Pulse Width Modulation}
    %
    A constant current switched on and off at a fixed frequency. The time the signal is considered high relatively to the
    period time is called the duty cycle.
    %
    \subsection*{Amplitude, Rise Time, Fall Time, Pulse Width}
    %
    Amplitude: In a wider sense a term used to characterize repeating phenomena. It is defined as a signals maximum deviation
    from its arithmetic mean value.\par
    Rise time: a technical term defined as the time it takes for a system to transition its output from \( 10\% \) to \( 90\% \)
    of an infinitesimal steep input signal.\par
    % Fall time:
    Pulse width: 
    %
    \subsection*{Bandwidth and Rise Time of an Oscilloscope}
    %
    Text
    %
\section{Preparation}
%
    \subsection*{Reflection on a Transmission Line}
    %
    \begin{figure}[h]
        \centering
        \begin{framed}
            \textbf{!!! Insert Diagram Here !!!}
        \end{framed}
    \end{figure}
    %
    \subsection*{SPICE-Simulation of a Boost Converter}
    %
    \begin{figure}[h]
        \centering
        \includegraphics[width=\textwidth]{Spice/circuit.jpg}
        \caption[Simulated circuit of a boost converter.]{Simulated circuit of a boost converter using \textsc{LTspice}.}
        \label{fig:simCircuit}
    \end{figure}
    %
    \begin{figure}[h]
        \centering\includegraphics[width=\textwidth]{Spice/plot.jpg}
        \caption{Plot of the output voltage at \textit{HV}. The voltage is subsequently progressing towards a peak voltage of \( \hat{U}_{HV} \approx \SI[]{150}[]{V} \) with rising PWM duty cycle.}
        \label{fig:plotSimCircuit}
    \end{figure}
    %
    \subsection*{Charge/Discharge Time of a Capacitor}
    %
    Charging:
    \begin{gather}
        U_{Br} = U_+ \left( 1 - e^{-\frac{t_{charge}}{R_6C_5}}\right) \nonumber \\
        \Leftrightarrow \nonumber \\
        t_{charge} = - \ln\left(1 - \frac{U_{Br}}{U_+}\right) \cdot R_6 C_5
        \label{eq:avalanche_charging_equation}
    \end{gather}
    Discharging:
    \begin{gather}
        U_{C_5} = U_{Br} \left(e^{-\frac{t_{discharge}}{R_7C_5}}\right) \nonumber \\
        \Leftrightarrow \nonumber \\
        t_{discharge} = -\ln\left( \frac{U_{C_5}}{U_{Br}} \right) \cdot R_7 C_5
        \label{eq:avalanche_discharging_equation}
    \end{gather}
    plugging in the values for \(U_{Br} = \SI{65}{V}, U_+ = \SI{75}{V}, U_{C_5} = \SI{5}{V}, C_5 = \SI{2.2}{pF}, R_6 = \SI{1}{M\ohm} \text{ and } R_7 = \SI{51}{\ohm}\)
    equates to the following charging/discharging times \(t_{charge}\) and \(t_{discharge}\):
    \begin{align}
        t_{charge} &= - \ln\left(1 - \frac{\SI{65}{V}}{\SI{75}{V}}\right) \cdot \SI{10^6}{\ohm} \cdot \SI{2.2 \cdot 10^{-12}}{F} \nonumber \\
        &\approx \SI{4.43 \cdot 10^{-6}}{s} \label{eq:avalanche_charging_time}\\
        \nonumber \\
        t_{discharge} &= -\ln\left( \frac{\SI{5}{V}}{\SI{65}{V}} \right) \cdot \SI{51}{\ohm} \cdot \SI{2.2 \cdot 10^{-12}}{F} \nonumber \\
        &\approx \SI{2.88 \cdot 10^{-10}}{s} \label{eq:avalanche_discharging_time}
    \end{align}
    With these numbers, the minimum time per charge/discharge cycle would be the sum of both times. Thus, the maximum number
    of repetitions per second \(f_{Rep}\) is
    \begin{equation}
        f_{Rep} = \left(t_{charge} + t_{discharge}\right)^{-1} \approx \SI{225.7}{kHz}
    \end{equation}
    %
    \subsection*{Cable Characteristics of RG-58/U Coaxial Cable}% A4
    %
    Nominal characteristic impedance: \(\SI{53}{\ohm}\)\par
    Nominal velocity of propagation: \(69.5\%\)\par
    Nominal delay (translates to the inverse of the absolute speed of propagation): \(\SI{4.85588}{\nicefrac{ns}{m}}\)\par
    The values above are taken from the technical data sheet \cite{Belden.RG-58/U.CoaxCable.Datasheet}.
    %
    \subsection*{Determining the Suitability of the Oscilloscope}% A5
    %

    %
    \subsection*{Sampling Rate}% A6
    %
    %